%
% Use the standard article template.
%
\documentclass{article}

% The geometry package allows for easy page formatting.
\usepackage{geometry}
\usepackage{float}
\geometry{letterpaper}


% Load up special logo commands.
\usepackage{doc}

% Package for formatting URLs.
\usepackage{url}

% Packages and definitions for graphics files.
\usepackage{graphicx}
\usepackage{epstopdf}
\DeclareGraphicsRule{.tif}{png}{.png}{`convert #1 `dirname #1`/`basename #1 .tif`.png}

%
% Set the title, author, and date.
%
\title{Headmaster Dream Design}
\author{Eric Dea}
\date{November 29, 2012}


%
% The document proper.
%
\begin{document}

% Add the title section.
\maketitle

% Add an abstract.
\abstract{
This paper explains my dream web interface for the headmaster web-app.  The headmaster web-app is an online database for a school that contains students info, events, grants, and other school related functions.  The paper will explain the top-level layout, a couple usage scenarios, reasons for the choices I make, and usability metrics.  These should effectively give the reader a good idea what my dream design is.  }

% Add various lists on new pages.
\pagebreak
\tableofcontents

\pagebreak
\listoffigures

\pagebreak

%
% Body text.
%
\section{Introduction}
\label{introduction}

In general, my dream design uses a mix of different interaction styles.  The choices for these styles has to do with the usability metrics in order to give the user the best experience using the web-app.  I do like the new interaction styles that involve the use of a computer's camera and speech.  Each interaction style has its own advantages and disadvantages.  Also, each user has a different mental model and experience with these interaction styles.  Thus, my dream design would like to incorporate the advantages of each interaction style to create a truly "perfect" interface.



\end{document}
