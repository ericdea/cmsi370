%
% Use the standard article template.
%
\documentclass{article}

% The geometry package allows for easy page formatting.
\usepackage{geometry}
\usepackage{float}
\geometry{letterpaper}


% Load up special logo commands.
\usepackage{doc}

% Package for formatting URLs.
\usepackage{url}

% Packages and definitions for graphics files.
\usepackage{graphicx}
\usepackage{epstopdf}
\DeclareGraphicsRule{.tif}{png}{.png}{`convert #1 `dirname #1`/`basename #1 .tif`.png}

%
% Set the title, author, and date.
%
\title{Headmaster Dream Design}
\author{Eric Dea}
\date{November 29, 2012}


%
% The document proper.
%
\begin{document}

% Add the title section.
\maketitle

% Add an abstract.
\abstract{
This paper explains my dream web interface for the headmaster web-app.  The headmaster web-app is an online database for a school that contains students info, events, grants, and other school related functions.  The paper will explain the top-level layout, a couple usage scenarios, reasons for the choices I make, and usability metrics.  These should effectively give the reader a good idea what my dream design is.  }

% Add various lists on new pages.
\pagebreak
\tableofcontents

\pagebreak
\listoffigures

\pagebreak

%
% Body text.
%
\section{Introduction}
\label{introduction}

In general, my dream design uses a mix of different interaction styles.  The choices for these styles has to do with the usability metrics in order to give the user the best experience using the web-app.  I do like the new interaction styles that involve the use of a computer's camera and speech.  Each interaction style has its own advantages and disadvantages.  Also, each user has a different mental model and experience with these interaction styles.  Thus, my dream design would like to incorporate the advantages of each interaction style to create a truly "perfect" interface.

\section{Top Level Design and Layout}

The design for my headmaster web-app is comprised of a login page, students page, grants page, events page, and a users page.

\subsection{Login Page}

For a new user or default settings, there is a standard form login design.  There are two fields, one for username and the other for password.  Inside the web-app, under the user settings, the user can edit their login preferences.  The preference options include a facial recognition login (built-in camera or attached camera through USB port), a speech recognition \cite{sphinx} login, the standard form filling login, or a combination of the three. Personally, I would use a combination of speech and facial recognition login. It provides a secure, and interesting way to login.  The form filling login is always visible to each user signing in, just in case the speech or facial recognition fail to work for some reason.

The background is default white, but can be changed in the user settings.  There will be a 'remember me' checkbox to save the username and background login image.  The facial recognition and speech login are available by default, but will not count as a valid login unless the user changes his/her preferences in the user preferences page.

\subsection{Welcome Page}

After the user logs in, the user will be brought to the welcome screen.  The page will have a 3D layout of a school (i.e. LMU), where the user can rotate, zoom, and click various sections of the school to bring up different pages.  The page would 'feel' like a virtual world where you see the school from a helicopter view. This layout uses direct manipulation to navigate around the virtual school.  The control are:

\begin{itemize}
\item Rotate: Click and drag the mouse.
\item Pan: Arrow Keys.
\item Zoom in/out: mouse scroll up/down.
\item Select buildings/links: Mouse click. 
\end{itemize}

The Student page link will be located in the dorm area as well as the admissions building.  The events page link is located in the Event Scheduling building.  The grants page link is located in the Financial aid office building.  The users page link is located in the registrat's office. 

There is also a button in the top right that brings a drop down menu showing the previous links all at once.  This design incorporates direct manipulation with forms and menus.  The direct manipulation gives a cool and fun feel to the program while the forms and menu dropdown gives a familiar more professional feel.

 % Generate the bibliography.
\bibliography{cmsi370-dreamDesign}
\bibliographystyle{unsrt}

\end{document}
