%
% Use the standard article template.
%
\documentclass{article}

% The geometry package allows for easy page formatting.
\usepackage{geometry}
\usepackage{float}
\geometry{letterpaper}


% Load up special logo commands.
\usepackage{doc}

% Package for formatting URLs.
\usepackage{url}

% Packages and definitions for graphics files.
\usepackage{graphicx}
\usepackage{epstopdf}
\DeclareGraphicsRule{.tif}{png}{.png}{`convert #1 `dirname #1`/`basename #1 .tif`.png}

%
% Set the title, author, and date.
%
\title{Skeuomorphic Design}
\author{Eric Dea}
\date{October 30, 2012}


%
% The document proper.
%
\begin{document}

% Add the title section.
\maketitle

% Add an abstract.
\abstract{
This paper explains how skeuomorphic interface designs fit the user's and developer's mental model.  Skeuomorphic designs are intended to create a familiar layout for users, especially new or novice users.  Effectively aligning user and developer mental models is exactly what skeuomorphism is about. When trying to create an effective usable interface, skeuomorphism comes to the mind of the developer.  This paper will show that the usability of skeuomorphic user interface designs is directly related to the successful alignment of the user and developer mental models.}

% Add various lists on new pages.
\pagebreak
\tableofcontents

\pagebreak
\listoffigures

\pagebreak

%
% Body text.
%
\section{Introduction}
\label{introduction}

A Skeuomorph, or Skeuomorphism, is when a product imitates design elements functionally necessary in the original product design, but that becomes ornamental in the new product design \cite{wiki}.  So in the computer world, a skeuomorphic user interface design is one that tries to imitate a 'real world' interface by using various decorations.  The website interface designer tries to replicate what the user's mental model is, what the user pictures in their mind.  This relationship between the user's mental model and the developer's mental model is key to creating a successful user interface.  In this paper, I hope to show how important this relationship is and how skeuomorphism directly relates to this relationship between the user and developer's mental models.


\section{Skeuomorphism}

In a User Interface, the elements of the interface can either be functional or non-functional.  The functional elements can either be skeuomorphs or just plain old elements that are necessary to the interface.  The non-functional elements are there for decoration and visual satisfaction.  This is where skeuomorphs come into play, for they create a better looking layout for most novice computer users.

\subsection{Non-Functional Elements}

Skeuomorphism, or skeuomorphs, have a certain abstract quality and property.  The layout and content of a user interface must have components that can be made abstract, or nonfunctional.  An abstract component can serve both as a building block for new development and as a reminder of its future functionality \cite{skeu-software}. So, even though most of these 'abstract' elements are non-functional and are just decorations, there is an option for future development to make these elements functional.  An example of this abstract element, or non-functional, is the rip/tear on Apple's iCal.

\begin{figure}[H]
\centering
\includegraphics[width=3.5in]{ical.jpeg} 

\caption{iCal with Skeuomorphism}
\label{iCal}
\end{figure}

If you look at the top of Figure \ref{iCal}, there are rip marks indicating that the user 'ripped' out the past month, just like an ordinary calendar.  The tear has no functionality, for it is just there for decoration.

\subsection{Functional Elements}

On the other side of Skeuomorphism, there are elements of a user interface that exactly replicate the real life version and also have functionality.  One common example of this is Figure \ref{traktor}, which is a DJ mixing software Traktor.

\begin{figure}[H]
\centering
\includegraphics[width=4.5in]{traktorUI.jpg} 

\caption{Traktor Scratch Pro 2}
\label{traktor}
\end{figure}

All of the buttons and knobs on the interface are fully functional.  A DJ mixing set has very similar, if not exact, buttons and knobs.  Because all of the knobs are turned just like a real DJ mixer, the skeuomorphic elements are functional.  A skeuomorphic user interface where all of its elements are functional is the perfect interface where the user's mental model and the developer's mental model is on the same page.  For example, say the user is a DJ who uses mixers but has not ever DJed on a computer.  If this user were to try out this digital interface, the transition is very easy for everything on his own mixer looks and works exactly the same on the computer.

\section{Mental Model}

A mental model is an explanation of someone's thought process about how something works in the real world \cite{wiki-mental}.  When trying to successfully align the mental models of the user and the developer, their thought processes must be similar.  Because skeuomorphism is the imatating of elements to a new product, the quality of the skeuomorphism is directly related to the developer's alignment of the mental model of the user.  If done well, skeuomorphism can improve the user interface by providing a familiar mental model to the user.

\subsection{Apple's view on Mental Model}
The user already has a mental model of any given software.  This mental model comes from real world experiences, experience with the software/interface and the experience with computers in general \cite{apple}.  When trying to create an interface, Apple suggests the follow characteristics:

\begin{itemize}
\item Familiarity
\item Simplicity
\item Availability
\item Discoverability 
\end{itemize}

These four characteristics are great guidelines for trying to fit the user's mental model.  Almost all of Apple's products reflect these characteristics, and whether they are good or not is determined by the user.  There are both positive and negative examples of using skeuomorphic interfaces when trying to capture the user's mental model.

\subsubsection{Positive Examples}

\subsubsection{Negative Examples}
Apple is a company that uses skeuomorphic designs a lot in their products. Such examples are the iCal, Notes (iPhone), folder icons to indicate folders, and iBooks/Newsstand.  While this might fit the mental model of people who are new to the world of computers, there are a lot of complaints by experienced users who feel that the skeuomorphic interface does not fit their mental model. 

\section{Conclusion}

Wrap up your paper with an ``executive summary'' of the paper itself, reiterating its subject and its major points.  If you want examples, just look at the conclusions from the literature.

% Generate the bibliography.
\bibliography{cmsi370-skeuomorphic}
\bibliographystyle{unsrt}

\end{document}
